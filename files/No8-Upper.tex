      \begin{MyInnerSplitBox}{Year 9 and above}
        The larger square has sides of length \SI{20}{\cm}, the smaller square has sides of length \SI{10}{\cm}. Find the shaded area.
      \iftoggle{SOLUTION}{%conditional output begin
      \begin{MySolutionBox}
        Label some vertices and add the line DR, perpendicular to MN and through D. EQ has length \(a\), as EC is \SI{20}{\cm}, PC must be \(20-a-10=10-a\).\par
        Now consider \(\bigtriangleup EMQ\) and \(\angle MEQ\).\par
        As ABCE and MNPQ are both squares, \(EC\parallel MN\).\par
        So \(\angle DMN = \angle MEQ\) and, \(\bigtriangleup EMQ\) and \(\bigtriangleup DMR\) are similar.\par
        Notice also that by the same reasoning \(\angle PCN\) and \(\angle RND\) are equal.\par
        Now, if \(MR = a\) then \(DR=10\) by similar triangles. Then if \(DR=10\) then \(RN = 10-a\) by similarity (congruency) with \(\bigtriangleup CNP\).\par
        So assuming \(MR=a\) leads to \(RN=10-a\) and we have \(MR+RN=a + (10-a) = 10\) which is correct.
      We have shown DR is \SI{10}{\cm}, so \(\bigtriangleup MEQ\) and \(\bigtriangleup DMN\) are congruent, they have the same area. Also \(\bigtriangleup PCN\) and \(\bigtriangleup DNR\) have the same area.\par
        The shaded area is \(a\times 10 + (10-a)\times 10 = \SI{100}{\square\cm}\)\par
        You can see this looks right if you imagine folding the shaded triangles into the \(10\times 10\) square along their boundaries with the square.\par
        We have shown that the position of the smaller square along the top of the larger square, does not affect the shaded area.\par
        % A longer way to answer this would be to say that point A is the origin \((0,0)\) and find the equation of the two lines ED and CD, then show that the y-coordinate of their point of intersection is \(10\).\par
      \end{MySolutionBox}
      }{}%conditional output end
        \tcblower
        \begin{tikzpicture}[scale=0.7,line join=bevel]
          \coordinate (A) at (0,0);
          \iftoggle{SOLUTION}{\node at (A) [left] {A};}{}
          \coordinate (B) at (6,0);
          \iftoggle{SOLUTION}{\node at (B) [right] {B};}{}
          \coordinate (C) at (6,6);
          \iftoggle{SOLUTION}{\node at (C) [right] {C};}{}
          \coordinate (D) at (2,12);
          \iftoggle{SOLUTION}{\node at (D) [left] {D};}{}
          \coordinate (E) at (0,6);
          \iftoggle{SOLUTION}{\node at (E) [left] {E};}{}
          \coordinate (M) at (1,9);
          \iftoggle{SOLUTION}{\node at (M) [left] {M};}{}
          \coordinate (N) at (4,9);
          \iftoggle{SOLUTION}{\node at (N) [right] {N};}{}
          \coordinate (P) at (4,6);
          \iftoggle{SOLUTION}{\node at (P) [below] {P};}{}
          \coordinate (Q) at (1,6);
          \iftoggle{SOLUTION}{\node at (Q) [below] {Q};}{}
          \fill[Khaki1] (E) -- (M) -- (Q);
          \fill[Khaki1] (M) -- (D) -- (N);
          \fill[Khaki1] (N) -- (C) -- (P);
          \draw[] (E) -- (C) -- (D) -- (E) -- (A) -- (B) -- (C);
          \draw[] (Q) -- (M) -- (N) -- (P);
          \iftoggle{SOLUTION}{
          \draw[dashed,gray] (D) -- (2,9) coordinate (R);
          \node at (R) [below] {R};
          \node at (3,0) [above] {\(20\)};
          \node at (0,3) [right] {\(20\)};
          \node at (1,7.5) [right] {\(10\)};
          \node at (2.5,6) [below] {\(10\)};
          \node at (0.5,6) [below=0mm,blue] {\(a\)};
          \node at (5,6) [below=-1mm,blue] {\(10-a\)};
          }{}
        \end{tikzpicture}
    \end{MyInnerSplitBox}

