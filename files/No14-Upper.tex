      \begin{MyInnerSplitBox}{Year 9 and above}
        The lengths of the sides of the three squares are consecutive integers. Find the area covered by the three squares.\par
      \iftoggle{SOLUTION}{%conditional output begin
      \begin{MySolutionBox}
        Let's call the length of the side of the smallest square \(a\).\par
        We can then write some of the other dimensions in terms of \(a\), and find the dimensions of the rectangular overlap of the larger two squares.\par
        Triangles \(\bigtriangleup ABC\) and \(\bigtriangleup ADE\) are congruent so \(ED=2\).\par
        Applying Pythagoras' Thm. to \(\bigtriangleup BDF\) we have,\par
        \begin{align*}
          4^{2} + (2a+2)^{2} &= (4\sqrt{10})^{2}\\
          4a^2 + 8a + 4 &= 160\\
          4a^{2} + 8a + 20 - 160 &= 0\\
          a^{2} + 2a - 35 &= 0\\
          (a+7)(a-5) &= 0
        \end{align*}
        Distance must be positive, so we reject the \(a=-7\) solution, and take \(a=5\).\par
        Now the total area is,
        \begin{align*}
          5^{2} + (5+1)^{2} + (5+2)^{2} &= 25 + 36 + 49\\
                                        &= 110\\
          \shortintertext{minus the overlapping rectangle,}
          110 - 2\times 1 &= \SI{108}{\square\cm}
        \end{align*}
      \end{MySolutionBox}
    }{}%conditional output end
        \tcblower
        \begin{tikzpicture}[scale=0.6]
          \draw[black,thick,fill=RosyBrown1] (0,0) coordinate (A) -- (6,0) coordinate (B) -- (6,6) coordinate (C) -- (0,6) coordinate (D) -- cycle;
          \draw[black,thick,fill=PaleTurquoise1] (B) -- (11,0) coordinate (E) -- (11,5) coordinate (F) -- (6,5) coordinate (G) -- cycle;
          \draw[black,thick,fill=NavajoWhite1,opacity=0.4] (4,5) coordinate (H) -- (F) -- (11,12) coordinate (I) -- (4,12) coordinate (J) -- cycle;
          \draw[Stealth-Stealth,Tomato4] (8,0) -- (J) node[pos=0.7,right] {\(4\sqrt{10}\)};
          \draw[dashed] (H) -- (4,0);
          \iftoggle{SOLUTION}{
            \node[below] (M) at (8.5,0) {\(a\)};
            \node[below] (N) at (2.5,0) {\(a+1\)};
            \node[above] (O) at (7.5,12) {\(a+2\)};
            \node[above] (P) at (2,6) {\(a-1\)};
            \node[below] (Q) at (5,5) {\(2\)};
            \node[left] (R) at (4,5.5) {\(1\)};
            \node[above right] at (C) {\(A\)};
            \node[above] at (J) {\(B\)};
            \node[below] at (7.7,0) {\(D\)};
            \node[above right] at (4,6) {\(C\)};
            \node[below] at (B) {\(E\)};
            \node[below] at (4,0) {\(F\)};
            \draw[dashed,Stealth-Stealth] (3.3,6) -- (3.3,12) node[midway,fill=white,inner sep=0mm] {\(a+1\)};
          }{}
        \end{tikzpicture}
    \end{MyInnerSplitBox}

