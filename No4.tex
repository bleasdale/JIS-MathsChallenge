% !TEX program = xelatex
\documentclass[a4paper]{article}
\usepackage{amsthm}
\usepackage{amssymb}
\usepackage{bm}
\usepackage{mathtools}
\usepackage[x11names]{xcolor}
\usepackage{xparse}
\usepackage{fontspec}
\usepackage{unicode-math}
\setromanfont{DovesType-Regular.otf}
\setsansfont{Andika}
\setmathfont{Asana Math}[Scale=1]

\usepackage{varwidth}
\usepackage{siunitx}
\usepackage{graphicx}
\usepackage[margin=1cm]{geometry}
\usepackage[most]{tcolorbox}
\usepackage{pgfplots}
\pgfplotsset{compat=newest}
\tcbuselibrary{skins,xparse,poster}
% \usetikzlibrary{fadings}
\usetikzlibrary{calc, plotmarks, shapes, shapes.geometric, positioning, angles, intersections, quotes, through, patterns, turtle, arrows.meta}
\usetikzlibrary{decorations.markings}
% \usepackage{etoolbox}
\usepackage{tkz-euclide}

%%%%%%%%%%%%%%%%%%%%%%%%%%%%%%%%%%%%%%%%%%%%%%%%%%%%%%%%%
\newcommand\markangle[9]{% origin X Y radius radiusmark mark colour opacity
%  % fill red circle offset-from-centre
  \begin{scope}
    \path[clip] (#1) -- (#2) -- (#3);
    \fill[color=#7,fill opacity=#8,draw=black,name path global=pcircle]  % global declaration required otherwise pcircle is not seen by the `named intersections=' lines below.
    (#1) circle (#4);
  \end{scope}
  % middle calculation
  \path[name path=line one] (#1) -- (#2);
  \path[name path=line two] (#1) -- (#3);
  \path[%
  name intersections={of=line one and pcircle, by={inter one}},
  name intersections={of=line two and pcircle, by={inter two}}
  ] (inter one) -- (inter two) coordinate[pos=#9] (place);
  % put mark
  \node at ($(#1)!#5!(place)$) {\scriptsize{#6}};
}

% \newcommand{\condSoln}[2]{\ifcsdef{r@#1}{#2}{}}

% \newcommand\fadingtext[3][]{%
%    \begin{tikzfadingfrompicture}[name=fading letter]
%      \node[text=transparent!0,inner xsep=0pt,outer xsep=0pt,#1] {#3};
%    \end{tikzfadingfrompicture}%
%    \begin{tikzpicture}[baseline=(textnode.base)]
%      \node[inner sep=0pt,outer sep=0pt,#1](textnode){\phantom{#3}};
%      \shade[path fading=fading letter,#2,fit fading=false]
%      (textnode.south west) rectangle (textnode.north east);%
%    \end{tikzpicture}%
% }

\definecolor{JISpurple}{RGB}{89,72,122}
\definecolor{JISivory}{RGB}{241,234,221}
\definecolor{JIStaupe}{RGB}{183,156,154}
\definecolor{PaleGreen}{RGB}{240,255,240} % 'Honeydew'

\AddToHook{shipout/background}{%
    \put (0in,-\paperheight){\includegraphics[width=\paperwidth,height=\paperheight]{images/R10V5-R.png}}%
}

\newtcolorbox{MyOuterBox}{%
  enhanced,
  frame style=JISpurple,
  colback=JISivory,
  colframe=JISpurple,
  title={\includegraphics[width=0.9cm,height=0.9cm]{images/JIS Final Logo FA-02.png}\raisebox{3mm}{\Large{Maths Challenge}\hspace{26em} \Large{\bfseries\sffamily 4}}},
}

\newtcolorbox{MyInnerBox}[2][]{enhanced,%empty,
coltitle=JISpurple,colback=white,
fonttitle=\bfseries\sffamily,
attach boxed title to top left={yshift=-1.5mm},
boxed title style={empty, size=small, top=1mm, bottom=0pt},
varwidth boxed title=0.5\linewidth,
frame code={
  \path (title.east|-frame.north) coordinate (aux);
\path[draw=JISpurple, line width=0.5mm, rounded corners,fill=white]
(frame.west) |- ([xshift=-2.5mm]title.north east) to[out=0, in=180] ([xshift=7.5mm]aux)-|(frame.east)|-(frame.south)-|cycle;
},
title={#2},#1}

\newtcolorbox{MyInnerSplitBox}[2][]{enhanced,%empty,
bicolor,sidebyside,sidebyside align=top seam,
righthand width=5.5cm,colbacklower=white,
coltitle=JISpurple,colback=white,
fonttitle=\bfseries\sffamily,
attach boxed title to top left={yshift=-1.5mm},
boxed title style={empty, size=small, top=1mm, bottom=0pt},
varwidth boxed title=0.5\linewidth,
frame code={
  \path (title.east|-frame.north) coordinate (aux);
\path[draw=JISpurple, line width=0.5mm, rounded corners,fill=white]
(frame.west) |- ([xshift=-2.5mm]title.north east) to[out=0, in=180] ([xshift=7.5mm]aux)-|(frame.east)|-(frame.south)-|cycle;
},
title={#2},#1}


\newtcolorbox{MySolutionBox}{%
  enhanced, frame style=JISpurple,
  colback=PaleGreen, colframe=green,
  title={\Large Solution},
  drop fuzzy shadow,
  halign=left,
}

%%%%%%%%%%%%%%%%%%%%%%%%%%%%%%%%%%%%%%%%%%%%%%%%%%
\newtoggle{SOLUTION}
%%% Uncomment the appropriate line below to show solutions %%%
\toggletrue{SOLUTION}
% \togglefalse{SOLUTION}
%%%%%%%%%%%%%%%%%%%%%%%%%%%%%%%%%%%%%%%%%%%%%%%%%


%%%%%%%%%%%%%%%%%%%%%%%%%%%%%%%%%%%%%%%%%%%%%%%%%%
%%%%%%            DOCUMENT BEGINS           %%%%%%
%%%%%%%%%%%%%%%%%%%%%%%%%%%%%%%%%%%%%%%%%%%%%%%%%%
\begin{document}


  \begin{MyOuterBox}
    \iftoggle{SOLUTION}{Here are the full, or partial solutions.
    }{
      Welcome to this week's Maths Challenge!\\
      Everyone is encouraged to take part.\\
      There are 3 to 5 merit points for correct solutions, depending on the difficulty of the problem and how impressed the marker is with your solution.\\
      There are house points for students who submit 7 or more correct solutions in one term.\\
      Solutions must be explained in detail, responses with just the answer may be ignored.\\
      Drop your solution in the box in the staffroom by Tuesday.
    }
       \begin{MyInnerBox}{Year 8 and below}
      Write a ten-digit number so that the first digit tells how many zeros there are in the number, the second how many ones, the third how many twos, and so forth. \(1210\) is a four digit example, there is \(1\) zero, \(2\) ones, \(1\) two and \(0\) threes.
      \iftoggle{SOLUTION}{%conditional output begin
      \begin{MySolutionBox}
        Numbers that behave this way are called "self-descriptive" numbers.\par
        Rather than trying out numbers at random, we start with a number and then correct it step by step to 'home in' on the solution. We start off trying nine zeroes and a one. Then we try correcting an error per line.\par
        \begin{tabular}{l||c|c|c|c|c|c|c|c|c|c|l}
          Digit & 0 & 1 & 2 & 3 & 4 & 5 & 6 & 7 & 8 & 9\\ \hline
          Try 0 & 0 & 0 & 0 & 0 & 0 & 0 & 0 & 0 & 0 & 1 & Wrong, there are 9 zeroes, but the zero place says zero.\\
          Try 1 & 9 & 0 & 0 & 0 & 0 & 0 & 0 & 0 & 0 & 0 & Wrong, now there's a nine but the nines place says zero.\\
          Try 2 & 9 & 0 & 0 & 0 & 0 & 0 & 0 & 0 & 0 & 1 & Wrong, there are 8 zeroes.\\
          Try 3 & 8 & 0 & 0 & 0 & 0 & 0 & 0 & 0 & 1 & 0 & Wrong, there's a one.\\
          Try 4 & 8 & 1 & 0 & 0 & 0 & 0 & 0 & 0 & 1 & 0 & Wrong, there are 2 ones.\\
          Try 5 & 8 & 2 & 1 & 0 & 0 & 0 & 0 & 0 & 1 & 0 & Wrong, the ones and twos are right but there are 6 zeroes.\\
          Try 6 & 6 & 2 & 1 & 0 & 0 & 0 & 1 & 0 & 0 & 0 & Correct!
        \end{tabular}\\
        How about we try a different starting number? Sometimes we can alter a couple of numbers each iteration.\par
        \begin{tabular}{l||c|c|c|c|c|c|c|c|c|c|l}
          Digit & 0 & 1 & 2 & 3 & 4 & 5 & 6 & 7 & 8 & 9\\ \hline
          Try 0 & 1 & 2 & 3 & 4 & 5 & 1 & 2 & 3 & 4 & 5 & Wrong, there are no nines.\\
          Try 1 & 1 & 2 & 3 & 4 & 5 & 1 & 2 & 3 & 4 & 0 & Wrong, there are no eights.\\
          Try 2 & 2 & 2 & 3 & 4 & 5 & 1 & 2 & 3 & 0 & 0 & Wrong, there are no sevens.\\
          Try 3 & 3 & 2 & 3 & 4 & 5 & 1 & 2 & 0 & 0 & 0 & Wrong, there are no sixes.\\
          Try 4 & 4 & 2 & 3 & 4 & 5 & 1 & 0 & 0 & 0 & 0 & Wrong, there are 2 fours, not five.\\
          Try 5 & 4 & 2 & 3 & 4 & 2 & 0 & 0 & 0 & 0 & 0 & Wrong, are \(0\) ones.\\
          Try 6 & 4 & 0 & 3 & 4 & 2 & 0 & 0 & 0 & 0 & 0 & Wrong, there are 6 zeroes.\\
          Try 7 & 6 & 0 & 3 & 4 & 1 & 0 & 0 & 0 & 0 & 0 & Wrong, there is 1 three.\\
          Try 8 & 6 & 0 & 3 & 1 & 0 & 0 & 1 & 0 & 0 & 0 & Wrong, there are 2 ones and no threes.\\
          Try 9 & 6 & 2 & 3 & 0 & 0 & 0 & 1 & 0 & 0 & 0 & Wrong, there's only 1 two.\\
          Try 10 & 6 & 2 & 1 & 0 & 0 & 0 & 1 & 0 & 0 & 0 & Correct!
        \end{tabular}\\
        Finally, we got to the same answer as before. In fact, you always end up with the same number, though sometimes you can end up in a repeating loop along the way.
      \end{MySolutionBox}
    }{}%conditional output end
    \end{MyInnerBox}


    \vspace{0.4cm}
          \begin{MyInnerBox}{Year 9 and above}
        Alice rides her bicycle up a certain hill at \(\SI{10}{\km\per\hour}\) and returns at \(\SI{20}{\km\per\hour}\). What is her average speed for the entire trip?
      \iftoggle{SOLUTION}{%conditional output begin
      \begin{MySolutionBox}
        Recall that average speed is the \emph{total distance} divided by the \emph{total time} taken to cover the distance. (You \emph{cannot} average the two given speeds!)\par
        Let \(A\) be the average speed for the whole journey and let \(d\) be the distance up (and therefore down) the hill.\\
        Let \(t_{u}\) and \(t_{d}\) be the times Alice takes to ride up and down the hill, respectively.
        \begin{alignat*}{3}
          \text{Up the hill:  }  10 &= \frac{d}{t_{u}}  & && \text{Down the hill:  }  20 &= \frac{d}{t_{d}}\\
          t_{u} &= \frac{d}{10} & && t_{d} &= \frac{d}{20}\\
          A &= \frac{total\, distance}{total\, time} = \frac{2d}{t_{u} + t_{d}}\\
            &= \frac{2d}{\frac{d}{10} + \frac{d}{20}} = \frac{2d}{\frac{2d+d}{20}} = \frac{40d}{3d}\\
            &= \frac{40}{3} = \SI[parse-numbers=false]{13.\overline{3}}{\km\per\hour}
        \end{alignat*}
      \end{MySolutionBox}
    }{}%conditional output end
    \end{MyInnerBox}


  \end{MyOuterBox}

%%%%%%%%%%%%%%%%%%%%%%%%%%%%%%%%%%%%%%%%%%%%%%%%%%
\end{document}



