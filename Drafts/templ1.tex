% !TEX program = xelatex
\documentclass[a4paper]{article}
\usepackage{amsthm}
\usepackage{amssymb}
\usepackage{bm}
\usepackage{mathtools}
\usepackage[x11names]{xcolor}
\usepackage{xparse}
\usepackage{fontspec}
\usepackage{unicode-math}
\setromanfont{DovesType-Regular.otf}
\setsansfont{Andika}
\setmathfont{Asana Math}[Scale=1]

\usepackage{varwidth}
\usepackage{siunitx}
\usepackage{graphicx}
\usepackage[margin=1cm]{geometry}
\usepackage[most]{tcolorbox}
\tcbuselibrary{skins,xparse,poster}
\usetikzlibrary{fadings}
\usepackage{etoolbox}
\newcommand{\condSoln}[2]{\ifcsdef{r@#1}{#2}{}}

\newcommand\fadingtext[3][]{%
   \begin{tikzfadingfrompicture}[name=fading letter]
     \node[text=transparent!0,inner xsep=0pt,outer xsep=0pt,#1] {#3};
   \end{tikzfadingfrompicture}%
   \begin{tikzpicture}[baseline=(textnode.base)]
     \node[inner sep=0pt,outer sep=0pt,#1](textnode){\phantom{#3}};
     \shade[path fading=fading letter,#2,fit fading=false]
     (textnode.south west) rectangle (textnode.north east);%
   \end{tikzpicture}%
}

\definecolor{JISpurple}{RGB}{89,72,122}
\definecolor{JISivory}{RGB}{241,234,221}
\definecolor{JIStaupe}{RGB}{183,156,154}
\definecolor{PaleGreen}{RGB}{240,255,240} % 'Honeydew'

\AddToHook{shipout/background}{%
    \put (0in,-\paperheight){\includegraphics[width=\paperwidth,height=\paperheight]{images/R10V5-R.png}}%
}

\newtcolorbox{MyOuterBox}{%
  enhanced,
  frame style=JISpurple,
  colback=JISivory,
  colframe=JISpurple,
  title={\includegraphics[width=0.9cm,height=0.9cm]{images/JIS Final Logo FA-02.png}\raisebox{3mm}{\Large{Maths Challenge}\hspace{26em} \Large{\bfseries\sffamily 1}}},
}

\newtcolorbox{MyInnerBox}[2][]{enhanced,%empty,
coltitle=JISpurple,colback=white,
fonttitle=\bfseries\sffamily,
attach boxed title to top left={yshift=-1.5mm},
boxed title style={empty, size=small, top=1mm, bottom=0pt},
varwidth boxed title=0.5\linewidth,
frame code={
  \path (title.east|-frame.north) coordinate (aux);
\path[draw=JISpurple, line width=0.5mm, rounded corners,fill=white]
(frame.west) |- ([xshift=-2.5mm]title.north east) to[out=0, in=180] ([xshift=7.5mm]aux)-|(frame.east)|-(frame.south)-|cycle;
},
title={#2},#1}

\newtcolorbox{MyInnerSplitBox}[2][]{enhanced,%empty,
bicolor,sidebyside,righthand width=4cm,colbacklower=white,
coltitle=JISpurple,colback=white,
fonttitle=\bfseries\sffamily,
attach boxed title to top left={yshift=-1.5mm},
boxed title style={empty, size=small, top=1mm, bottom=0pt},
varwidth boxed title=0.5\linewidth,
frame code={
  \path (title.east|-frame.north) coordinate (aux);
\path[draw=JISpurple, line width=0.5mm, rounded corners,fill=white]
(frame.west) |- ([xshift=-2.5mm]title.north east) to[out=0, in=180] ([xshift=7.5mm]aux)-|(frame.east)|-(frame.south)-|cycle;
},
title={#2},#1}


\newtcolorbox{MySolutionBox}{%
  enhanced, frame style=JISpurple,
  colback=PaleGreen, colframe=green,
  title={\Large Solution},
  drop fuzzy shadow,
}

%%%%%%%%%%%%%%%%%%%%%%%%%%%%%%%%%%%%%%%%%%%%%%%%%%
%%%%%%            DOCUMENT BEGINS           %%%%%%
%%%%%%%%%%%%%%%%%%%%%%%%%%%%%%%%%%%%%%%%%%%%%%%%%%
\begin{document}

%%%%%%%%%%%%%%%%%%%%%%%%%%%%%%%%%%%%%%%%%%%%%%%%%%
%%% Uncomment the line below to show solutions %%%
\label{sec:Solution}
%%%%%%%%%%%%%%%%%%%%%%%%%%%%%%%%%%%%%%%%%%%%%%%%%%

  \begin{MyOuterBox}
    \begin{MyInnerSplitBox}{Year 8 and below}
      You have a chess board from which two diagonally opposite squares have been removed. You also have thirty-one dominoes, each of which can cover two squares of the chess board. Can the dominoes be arranged so that they cover all sixty-two squares of the chess board? If so, how? If not, why not?
      \condSoln{sec:Solution}{%conditional output begin
      \begin{MySolutionBox}
        You can try proving that the thirty-one dominoes will cover the sixty-two squares but it is easier to find a solution if you start by trying to show the puzzle cannot be solved.\\
      Notice that two diagonally opposite squares are always the same colour, and that a domino must always cover two squares of \emph{different} colours.\\
      Therefore the problem cannot be solved because there will be too few black squares.
      \end{MySolutionBox}
    }%conditional output end
      \tcblower
      \includegraphics[width=\linewidth]{images/bwchess-2.jpg}%
    \end{MyInnerSplitBox}
    \begin{MyInnerBox}{Year 9 and above}
      Find the integer solutions to this pair of equations.
      \begin{align}
        ab+c &= 2020\\
        a+bc &= 2021
      \end{align}
      \condSoln{sec:Solution}{%conditional output begin
      \begin{MySolutionBox}
        Subtract the first equation from the second.
        \begin{align*}
          a + bc - ab - c &= 1\\
          a(1-b) - c(1-b) &= 1\\
          (a-c)(1-b) &= 1\\
          \intertext{Notice that for the last line to be true, either (Case 1),}
          a-c = 1, \text{ and } 1-b &= 1\\
          \shortintertext{or (Case 2),}
          a-c = -1, \text{ and } 1-b &= -1\\
          \shortintertext{Case 1}
          \intertext{Since \(1-b = 1\) we have \(b=0\). Substituting this into the original two equations (1) \& (2), we have one set of solutions:}
          a &= 2021\\
          b &= 0\\
          c &= 2020
          \shortintertext{Case 2}
          \intertext{Since \(1-b = -1\) we have \(b = 2\). Substituting this into the original two equations we get a pair of simultaneous equations:}
          2a + c &= 2020\\
          a + 2c &= 2021
          \intertext{Solving by elimination or substitution we obtain the second set of solutions:}
          a &= 673\\
          b &= 2\\
          c &= 674
        \end{align*}
      \end{MySolutionBox}
    }%conditional output end
    \end{MyInnerBox}
  \end{MyOuterBox}
\end{document}



