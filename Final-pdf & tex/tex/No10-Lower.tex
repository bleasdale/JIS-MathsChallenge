   \begin{MyInnerBox}{Year 8 and below}
      Some molten rhenium is being poured into three containers, A, B and C. Container A gets 1 litre plus one third of what is left. Container B then receives 6 litres plus one third of what remains, then Container C gets the rest which is 40 litres. What is the volume of rhenium poured into Container B?
      \iftoggle{SOLUTION}{%conditional output begin
      \begin{MySolutionBox}
        Let \(R\) be the total volume of rhenium in litres.\par
        \begin{tabular}{c|c|c|c|c|l}
          Step & A & B & C & Remainder&\\\hline
          1 & \(1 + \frac{R-1}{3}\) & ? & \(40\) & \(R-\left(1+\frac{R-1}{3}\right)\) & Simplify the remainder,\\[3mm]
            & & & & \(\frac{2R-2}{3}\) &  simplify A and work out B in Step 2:\\
          2 & \(\frac{R+2}{3}\) & \(6+\frac{\frac{2R-2}{3}-6}{3}\) & \(40\) & \(0\) & Simplify:\\[3mm]
          3 & \(\frac{R+2}{3}\) & \(6+\frac{2R-20}{9}\) & \(40\) & \(0\) & Now we can equate the sum of A, B \& C to R.\\
        \end{tabular}
        \begin{align*}
          R &= \frac{R+2}{3} + 6 + \frac{2R-20}{9} + 40\\
          R &= \frac{3R+6}{9} + \frac{2R-20}{9} + 46\\
          R &= \frac{5R-14}{9} + 46\\
          9R - 5R &= -14 + 414\\
          4R &= 400\\
          R &= \SI{100}{\litre}
        \end{align*}
        Alternatively, we can attack the problem from the end, begin with Container C with \(40\) and work backwards:\par
        We know that C is \(40\) and from the question, this is \(\frac{2}{3}\) of the quantity that B got \(\frac{1}{3}\) of. So \(40\times\frac{1}{2}=20\).\par
        But we know that B got this plus \(6\), so B got \(26\). Then for Container A: it has \(\frac{1}{2}\) of \(66\), (Container B and Container C's volume) plus \(1\) more litre, that is \(34\) litres. There is less algebra this way but perhaps it is trickier to think about.\par
        So the total volume of rhenium is \(34 + 26 + 40 = \SI{100}{\litre}\).
      \end{MySolutionBox}
    }{}%conditional output end
    \end{MyInnerBox}

