\begin{MyInnerSplitBox}{Year 9 and above}
  Find the shaded area.\par
  \iftoggle{SOLUTION}{%conditional output begin
    \begin{MySolutionBox}
      There doesn't seem to be enough information given to be able to solve this! Yet\dots\par
      Shaded area \(A\)= area of the quadrant - area of the circle.
      \begin{align*}
        A &= \frac{\pi R^{2}}{4} - \pi r^{2}\\
        \shortintertext{In \(\bigtriangleup BDF\), we can apply Pythagoras' Thm.:}
        R^{2} &= (2r)^{2} + 12^{2}\\
        r^{2} &= \frac{R^{2}-144}{4}\\
        \shortintertext{Substituting:}
        A &= \frac{\pi R^{2}}{4} - \pi \frac{R^{2}-144}{4}\\
        A &= \frac{\pi R^{2}}{4} - \frac{\pi R^{2}}{4} + 36\pi\\
        A &= 36\pi \approx \SI{113.097}{\square\cm}
      \end{align*}
      The unknowns just vanish, isn't that marvellous?! Notice the radius of the shaded quadrant is not fixed, as long as the chord tangent to the inner circle is \(\SI{12}{\cm}\), \(R\) and \(r\) change so that the shaded area remains \(36\pi\).\par
    \end{MySolutionBox}
  }{}%conditional output end
  \tcblower
  \begin{tikzpicture}[scale=1]
    \draw[black,fill=Plum2] (6,1) coordinate (C) -- (0,1) coordinate (B) -- (0,7) coordinate (A) to[bend left=45] (C);
    \draw[black,fill=white] (3,3) coordinate (E) circle (2);
    \draw[black] (0,5) coordinate (F) -- ({sqrt(20)},5) coordinate (D) node[midway,above] {\(\SI{12}{\cm}\)};
    \tkzMarkRightAngle[draw=black,size=0.15](B,F,D);
    \iftoggle{SOLUTION}{
      \draw[gray] (B) -- (D) node[midway,above,rotate=45,black] {\(R\)};
      \draw[gray] (E) -- (3,1) node[midway,right,black] {\(r\)};
      \draw[gray,dashed,Stealth-Stealth] (-0.6,1) -- (-0.6,5) node[black,midway,inner sep=0,fill=white] {\(2r\)};
      % \draw[gray,dashed,Stealth-Stealth] (-0.6,5) -- (-0.6,7) node[black,midway,inner sep=0,fill=white] {\(R-2r\)};
      \node[below left] at (B) {\(B\)};
      \node[above right] at (D) {\(D\)};
      \node[left] at (F) {\(F\)};
    }{}
  \end{tikzpicture}
\end{MyInnerSplitBox}

