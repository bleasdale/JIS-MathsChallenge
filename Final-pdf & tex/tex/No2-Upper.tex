      \begin{MyInnerBox}{Year 9 and above}
        A chauffeur always arrives at the train station at exactly five o'clock to pick up his boss and drive her home. One day his boss arrives an hour early, starts walking home, and is eventually picked up. She arrives home twenty minutes earlier than usual. How long did she walk before she met her chauffeur?\\
        Assume the chauffeur always drives at the same speed.
      \iftoggle{SOLUTION}{%conditional output begin
      \begin{MySolutionBox}
        This is difficult to solve if we start considering \(speed = \frac{distance}{time}\) for the walking and being driven parts of the boss's journey. Instead, consider the chauffeur. He departs to collect his boss at the same time as normal and would have arrived at 5 p.m. at the station. We are told, however, that the boss arrives home twenty minutes earlier than usual, and therefore, so does the chauffeur. In other words the chauffeur has had twenty minutes cut from his normal journey time. So his boss's walk saved him ten minutes travelling from the point at which he met his boss walking along the road to the station, and another ten minutes from the station back to the meeting point. So he must have met her walking along the road at 4:50 p.m. As the boss started walking home from the station at 4 p.m., she must have been walking for 50 minutes.
      \end{MySolutionBox}
    }{}%conditional output end
    \end{MyInnerBox}

