      \begin{MyInnerBox}{Year 9 and above}
        You have a four-digit, positive integer. Now you remove one of the four digits. The three digits that are left, in their original order from the four digit number, make a three-digit number. The sum of the four-digit number and the three-digit number is \(6031\). What is the four-digit number?\par
      \iftoggle{SOLUTION}{%conditional output begin
      \begin{MySolutionBox}
        We'll begin by describing the four-digit number as \(abcd\). The three digit number could be \(bcd\), removing the \(a\), \(acd\), removing the \(b\), \(abd\), or \(abc\). The additions for each of these cases look like:\par
        \underline{Case 1: a removed}\qquad\quad\underline{Case 2: b removed}\quad\qquad\underline{Case 3: c removed}\qquad\quad\underline{Case 4: d removed}\par
        \noindent\ttfamily
        \opadd[voperator=bottom,carryadd=false,operandstyle.1.1=\hole{d},
        operandstyle.1.2=\hole{c},
        operandstyle.1.3=\hole{b},
        operandstyle.1.4=\hole{a},
        operandstyle.2.1=\hole{d},
        operandstyle.2.2=\hole{c},
        operandstyle.2.3=\hole{b}]{5483}{548}\qquad\qquad\qquad
        \opadd[voperator=bottom,carryadd=false,operandstyle.1.1=\hole{d},
        operandstyle.1.2=\hole{c},
        operandstyle.1.3=\hole{b},
        operandstyle.1.4=\hole{a},
        operandstyle.2.1=\hole{d},
        operandstyle.2.2=\hole{c},
        operandstyle.2.3=\hole{a}]{5483}{548}\qquad\qquad\qquad
        \opadd[voperator=bottom,carryadd=false,operandstyle.1.1=\hole{d},
        operandstyle.1.2=\hole{c},
        operandstyle.1.3=\hole{b},
        operandstyle.1.4=\hole{a},
        operandstyle.2.1=\hole{d},
        operandstyle.2.2=\hole{b},
        operandstyle.2.3=\hole{a}]{5483}{548}\qquad\qquad\qquad
        \opadd[voperator=bottom,carryadd=false,operandstyle.1.1=\hole{d},
        operandstyle.1.2=\hole{c},
        operandstyle.1.3=\hole{b},
        operandstyle.1.4=\hole{a},
        operandstyle.2.1=\hole{c},
        operandstyle.2.2=\hole{b},
        operandstyle.2.3=\hole{a}]{5483}{548}\qquad\par
        \normalfont
        But notice that in the first three cases, the answer has a \(1\) in the units place, each obtained from \(d+d\). This is not possible, \(2d\) must be even, so the three-digit number must be \(abc\) as in the fourth case.\par
        Now, let's think about the value of \(a\). It could be \(6\), (no carry from the hundreds column), or it could be \(5\), (carry \(1\) from the hundreds colum.\par
        \underline{Case 1: \(a=6\)}\qquad\:   \underline{Case 2: \(a=5\)}\par
        \ttfamily
        \opadd[voperator=bottom,carryadd=false,
        operandstyle.1.1=\hole{d},
        operandstyle.1.2=\hole{c},
        operandstyle.1.3=\hole{b},
        operandstyle.1.4=\hole{6},
        operandstyle.2.1=\hole{c},
        operandstyle.2.2=\hole{b},
        operandstyle.2.3=\hole{6}]{5483}{548}\qquad\qquad
        \opadd[voperator=bottom,carryadd=true,
        carrystyle.4=\scriptsize\blue,
        carrystyle.3=\scriptsize\hole{},
        carrystyle.2=\scriptsize\hole{},
        operandstyle.1.1=\hole{d},
        operandstyle.1.2=\hole{c},
        operandstyle.1.3=\hole{b},
        operandstyle.1.4=\hole{5},
        operandstyle.2.1=\hole{c},
        operandstyle.2.2=\hole{b},
        operandstyle.2.3=\hole{5}]{5483}{548}\par
        \normalfont
        We can see that \(a=6\) will not work; if there is no carry, what value of \(b\) plus \(6\) can give zero in the hundreds place? None. So the case \(a=5\) must be correct. So now looking at Case 2, we must find the value of \(b\) such that \(b+5=10\), giving a \(0\) in the hundreds' answer column. So now we have two cases again, \(b=5\) (with no carry from the tens column), or \(b=4\) with \(1\) carried from the tens column:\par
        \underline{Case 1: \(b=5\)}\qquad\:   \underline{Case 2: \(b=4\)}\par
        \ttfamily
        \opadd[voperator=bottom,carryadd=true,
        carrystyle.4=\scriptsize\blue,
        carrystyle.3=\scriptsize\hole{},
        carrystyle.2=\scriptsize\hole{},
        operandstyle.1.1=\hole{d},
        operandstyle.1.2=\hole{c},
        operandstyle.1.3=\hole{5},
        operandstyle.1.4=\hole{5},
        operandstyle.2.1=\hole{c},
        operandstyle.2.2=\hole{5},
        operandstyle.2.3=\hole{5}]{5483}{548}\qquad\qquad
        \opadd[voperator=bottom,carryadd=true,
        carrystyle.4=\scriptsize\blue,
        carrystyle.3=\scriptsize\blue,
        carrystyle.2=\scriptsize\hole{},
        operandstyle.1.1=\hole{d},
        operandstyle.1.2=\hole{c},
        operandstyle.1.3=\hole{4},
        operandstyle.1.4=\hole{5},
        operandstyle.2.1=\hole{c},
        operandstyle.2.2=\hole{4},
        operandstyle.2.3=\hole{5}]{5483}{548}\par
        \normalfont
        In Case 1, in the tens column we must have \(c+5=3\), and yet there can be no carry. This is not possible so Case 1 is wrong, Case 2 is correct. So \(b=4\), and we must find \(c+4=3\). So \(c\) could be \(9\) with no carry from the ones column or \(c\) could be \(8\) with a carry from the ones column:\par
        \underline{Case 1: \(c=9\)}\qquad\:   \underline{Case 2: \(c=8\)}\par
        \ttfamily
        \opadd[voperator=bottom,carryadd=true,
        carrystyle.4=\scriptsize\blue,
        carrystyle.3=\scriptsize\blue,
        carrystyle.2=\scriptsize\hole{},
        operandstyle.1.1=\hole{d},
        operandstyle.1.2=\hole{9},
        operandstyle.1.3=\hole{4},
        operandstyle.1.4=\hole{5},
        operandstyle.2.1=\hole{9},
        operandstyle.2.2=\hole{4},
        operandstyle.2.3=\hole{5}]{5483}{548}\qquad\qquad
        \opadd[voperator=bottom,carryadd=true,
        carrystyle.4=\scriptsize\blue,
        carrystyle.3=\scriptsize\blue,
        carrystyle.2=\scriptsize\blue,
        operandstyle.1.1=\hole{d},
        operandstyle.1.2=\hole{8},
        operandstyle.1.3=\hole{4},
        operandstyle.1.4=\hole{5},
        operandstyle.2.1=\hole{8},
        operandstyle.2.2=\hole{4},
        operandstyle.2.3=\hole{5}]{5483}{548}\par
        \normalfont
        Nearly there! If \(c=9\) then in the ones column we would have \(d+9=1\) which means that \(d=2\) and there must be a carry, but Case 1 has no carry to the tens place so Case 1 is impossible. Therefore Case 2 with \(c=8\) must be true. But if \(c=8\) and we do have a carry to the tens column, then \(d\) has to be \(3\).\par
        \ttfamily
        \opadd[voperator=bottom,carryadd=true,
        carrystyle.4=\scriptsize\blue,
        carrystyle.3=\scriptsize\blue,
        carrystyle.2=\scriptsize\blue,
        operandstyle.1.1=\hole{3},
        operandstyle.1.2=\hole{8},
        operandstyle.1.3=\hole{4},
        operandstyle.1.4=\hole{5},
        operandstyle.2.1=\hole{8},
        operandstyle.2.2=\hole{4},
        operandstyle.2.3=\hole{5}]{5483}{548}\par
        \normalfont
        The four digit number is \(5483\).
      \end{MySolutionBox}
    }{}%conditional output end
    \end{MyInnerBox}

