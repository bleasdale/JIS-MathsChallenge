      \begin{MyInnerBox}{Year 10 and above}
       If \[ \sin^{3}\theta + \cos^{3}\theta = \frac{11}{16}\] find the value of \[\sin\theta + \cos\theta\]\par
       Hint: Factorise the first equation. Let \(x = \sin\theta + \cos\theta\). You should obtain a cubic polynomial in \(x\). \(x = \frac{1}{2}\) is a root.
      \iftoggle{SOLUTION}{%conditional output begin
        \vspace{0cm}
      \begin{MySolutionBox}
        Useful identities to start with are \(a^{3}+b^{3} = (a+b)(a^{2}-ab+b^{3})\) and \(\sin^{2}\theta + \cos^{2}\theta = 1\). Using these we have,
        \begin{align*}
          \sin^{3}\theta + \cos^{3}\theta = \frac{11}{16} &= (\sin\theta + \cos\theta)(\sin^{2}\theta - \sin\theta\cos\theta + \cos^{2}\theta)\\
          \frac{11}{16}  &= (\sin\theta + \cos\theta)(1-\sin\theta\cos\theta) \numberthis \label{eq1}\\
          \text{Let } x &= \sin\theta + \cos\theta \implies x^{2} = \sin^{2}\theta + \cos^{2}\theta + 2\sin\theta\cos\theta\\
          x^{2} &= 1 + 2\sin\theta\cos\theta \implies \frac{x^{2}-1}{2} = \sin\theta\cos\theta\\
          \shortintertext{Substituting in \eqref{eq1}, we obtain a cubic in \(x\):}
          \frac{11}{16} &= x\left(1-\frac{x^{2}-1}{2}\right) = x\left(\frac{3-x^{2}}{2}\right)\\
          11 &= 8x(3-x^{2})\\
          0 &= 8x^{3} -24x + 11\\
          \intertext{By the rational root theorem possible rational zeroes of this cubic are:}
          \pm 1,\pm 11,\pm\frac{1}{2},\pm\frac{11}{2},\pm\frac{1}{4},\pm\frac{11}{4},\pm\frac{1}{8},\pm\frac{11}{8}\\
          \shortintertext{Upon checking (and given in the hint!), we find that \(\frac{1}{2}\) is a root, so we can factorise:}
          0 &= (2x-1)(\_x^{2} + \_x + \_) = 8x^{3} - 24x + 11\\
          \intertext{where the underscores are values to find. The coefficient of \(x^{2}\) must be \(4\) and we can see the constant term must be \(-11\).}
          0 &= (2x-1)(4x^{2} +\_x -11) = 8x^{3} - 24x + 11\\
          \intertext{Now the coefficent of \(x\) must be \(2\) giving}
          0 &= (2x-1)(4x^{2} + 2x -11)\\
          \shortintertext{Solving the quadratic:}
          x &= \frac{1}{2},\frac{-1\pm 3\sqrt{5}}{4} \text{ that is } x = 0.5,\;1.42705\dots,\; -1.92705\dots\\
          \shortintertext{Our possible solutions are:}
          \sin\theta + \cos\theta &= 0.5,\;1.42705\dots,\; -1.92705\dots
        \shortintertext{However, the maximum and minimum values of \(\sin\theta + \cos\theta\) are \(\pm\sqrt{2}\), that is,}
          -\sqrt{2} &\leq \sin\theta + \cos\theta \leq \sqrt{2}\\
       \shortintertext{The two surd roots are out of this range,}
          \frac{-1 + 3\sqrt{5}}{4} > \sqrt{2},&\qquad \frac{-1 - 3\sqrt{5}}{4} < -\sqrt{2}
     \shortintertext{So the only possible solution is}
        \sin\theta + \cos\theta &= \frac{1}{2}
      \end{align*}
      \end{MySolutionBox}
    }{}%conditional output end
    \end{MyInnerBox}

